\documentclass{article}

% if you need to pass options to natbib, use, e.g.:
%     \PassOptionsToPackage{numbers, compress}{natbib}
% before loading neurips_2025
\PassOptionsToPackage{numbers,sort&compress}{natbib}

% The authors should use one of these tracks.
% Before accepting by the NeurIPS conference, select one of the options below.
% 0. "default" for submission
% \usepackage{neurips_2025}
% Using "final" option to show author names (not for actual submission)
 \usepackage[final]{neurips_2025}
% the "default" option is equal to the "main" option, which is used for the Main Track with double-blind reviewing.
% 1. "main" option is used for the Main Track
%  \usepackage[main]{neurips_2025}
% 2. "position" option is used for the Position Paper Track
%  \usepackage[position]{neurips_2025}
% 3. "dandb" option is used for the Datasets & Benchmarks Track
 % \usepackage[dandb]{neurips_2025}
% 4. "creativeai" option is used for the Creative AI Track
%  \usepackage[creativeai]{neurips_2025}
% 5. "sglblindworkshop" option is used for the Workshop with single-blind reviewing
 % \usepackage[sglblindworkshop]{neurips_2025}
% 6. "dblblindworkshop" option is used for the Workshop with double-blind reviewing
%  \usepackage[dblblindworkshop]{neurips_2025}

% After being accepted, the authors should add "final" behind the track to compile a camera-ready version.
% 1. Main Track
 % \usepackage[main, final]{neurips_2025}
% 2. Position Paper Track
%  \usepackage[position, final]{neurips_2025}
% 3. Datasets & Benchmarks Track
 % \usepackage[dandb, final]{neurips_2025}
% 4. Creative AI Track
%  \usepackage[creativeai, final]{neurips_2025}
% 5. Workshop with single-blind reviewing
%  \usepackage[sglblindworkshop, final]{neurips_2025}
% 6. Workshop with double-blind reviewing
%  \usepackage[dblblindworkshop, final]{neurips_2025}
% Note. For the workshop paper template, both \title{} and \workshoptitle{} are required, with the former indicating the paper title shown in the title and the latter indicating the workshop title displayed in the footnote.
% For workshops (5., 6.), the authors should add the name of the workshop, "\workshoptitle" command is used to set the workshop title.
% \workshoptitle{WORKSHOP TITLE}

% "preprint" option is used for arXiv or other preprint submissions
 % \usepackage[preprint]{neurips_2025}

% to avoid loading the natbib package, add option nonatbib:
%    \usepackage[nonatbib]{neurips_2025}

\usepackage[utf8]{inputenc} % allow utf-8 input
\usepackage[T1]{fontenc}    % use 8-bit T1 fonts
\usepackage{hyperref}       % hyperlinks
\usepackage{url}            % simple URL typesetting
\usepackage{booktabs}       % professional-quality tables
\usepackage{graphicx}       % figures
\usepackage{amsmath}        % math
\usepackage{amssymb}        % math symbols
\usepackage{amsfonts}       % blackboard math symbols
\usepackage{nicefrac}       % compact symbols for 1/2, etc.
\usepackage{microtype}      % microtypography
\usepackage{xcolor}         % colors
\usepackage[linesnumbered,ruled,vlined]{algorithm2e} % algorithm-style pseudocode

% Make algorithm2e look closer to common ML papers (SpecInfer/NeurIPS style)
\SetAlgoLined
\DontPrintSemicolon
\SetKwInput{KwIn}{Input}
\SetKwInput{KwOut}{Output}
\SetKwComment{Comment}{$\triangleright$\ }{}
\SetKwProg{Fn}{Function}{:}{end}
\SetKwFunction{FDraft}{DRAFTTREE}
\SetKwFunction{FVerify}{VERIFYTREE}
\SetKwFunction{FSelect}{SELECTCOMMIT}
\SetKwFunction{FUpdate}{UPDATECACHE}
\SetKwFunction{FFlatten}{FLATTENMASK}

% Note. For the workshop paper template, both \title{} and \workshoptitle{} are required, with the former indicating the paper title shown in the title and the latter indicating the workshop title displayed in the footnote. 
\title{DynaTree: Dynamic Tree-based Speculative Decoding with Adaptive Pruning for Efficient LLM Inference}


% The \author macro works with any number of authors. There are two commands
% used to separate the names and addresses of multiple authors: \And and \AND.
%
% Using \And between authors leaves it to LaTeX to determine where to break the
% lines. Using \AND forces a line break at that point. So, if LaTeX puts 3 of 4
% authors names on the first line, and the last on the second line, try using
% \AND instead of \And before the third author name.


\author{%
  Nuoyan Chen$^*$ \quad
  Jiamin Liu$^*$ \quad
  Zhaocheng Li\thanks{Equal contribution.} \\
  School of Computer Science\\
  Shanghai Jiao Tong University\\
  \texttt{\{cny123222, logic-1.0, lzc050419\}@sjtu.edu.cn}
}


\begin{document}


\maketitle


\begin{abstract}
Autoregressive decoding in large language models (LLMs) is fundamentally sequential and therefore underutilizes modern accelerator parallelism during token generation. Speculative decoding mitigates this bottleneck by letting a lightweight draft model propose multiple tokens that are verified in parallel by the target model; however, common linear variants explore only a single draft chain per step and can waste substantial computation when early tokens are rejected. We propose \textbf{DynaTree}, a tree-based speculative decoding framework that drafts multiple candidate continuations via top-$k$ branching and verifies the resulting token tree in one forward pass using tree attention. To control the exponential growth of the draft tree, DynaTree applies adaptive pruning that removes low-probability branches under an explicit node budget. Experiments on Pythia models show that DynaTree improves decoding throughput by up to 1.62$\times$ over standard autoregressive generation and consistently outperforms strong speculative decoding baselines across generation lengths.
\end{abstract}


\section{Introduction}

Large language models (LLMs) are typically deployed with autoregressive decoding, where each output token is generated after conditioning on all previously generated tokens. While transformer inference can exploit parallelism during the prefill stage, the decode stage remains inherently sequential and requires a full forward pass per token, leading to poor hardware utilization and high latency~\cite{flashattention,vllm}.

Speculative decoding alleviates this bottleneck by separating \emph{proposal} and \emph{verification}~\cite{leviathan2023fast}. A small draft model proposes several candidate tokens, and the target model verifies them in parallel; when the proposal matches the target distribution, multiple tokens can be committed per iteration. Importantly, with rejection sampling, speculative decoding preserves the exact output distribution of the target model~\cite{decoding_speculative}.

In practice, most speculative decoding systems employ \emph{linear} drafting: the draft model proposes a single chain of $K$ tokens. This design is brittle under draft--target mismatch: a rejection at an early position forces all subsequent drafted tokens to be discarded, wasting both draft computation and target-model verification work, and constraining achievable speedups~\cite{decoding_speculative}.

We argue that the single-path constraint is unnecessary. When multiple plausible next tokens compete, exploring several continuations in parallel increases the chance that at least one path aligns with the target model, thereby improving the expected number of accepted tokens per verification step. This motivates \emph{tree-based} speculation, where the draft expands multiple candidates via top-$k$ branching and the target verifies the resulting token tree with a structured, causality-preserving attention mask.

The central challenge is controlling verification cost: naive tree expansion grows exponentially with depth and branching. We present \textbf{DynaTree}, a tree-based speculative decoding framework with a lightweight adaptive pruning mechanism that removes low-probability branches while enforcing an explicit node budget. Empirically, DynaTree achieves up to 1.62$\times$ throughput improvement over standard autoregressive decoding on Pythia models and consistently outperforms strong speculative decoding baselines. In summary, our contributions are: (i) a practical tree-based speculative decoding algorithm with efficient tree attention verification; (ii) an adaptive pruning strategy that stabilizes the depth--breadth trade-off under a fixed verification budget; and (iii) an extensive empirical study characterizing these trade-offs across generation lengths.


\section{Related Work}

\subsection{Speculative Decoding}

Speculative decoding accelerates autoregressive generation by decoupling \emph{proposal} and \emph{verification}: a lightweight draft model proposes multiple tokens, and the target model verifies these candidates in parallel while preserving the exact output distribution through rejection sampling~\cite{leviathan2023fast,decoding_speculative}. Empirical and theoretical analyses highlight that achievable speedups depend critically on the acceptance behavior induced by the draft--target mismatch, and on the additional overhead introduced by drafting and verification~\cite{decoding_speculative,draft_tradeoff}. Recent systems-level studies further emphasize robustness challenges across heterogeneous request distributions and long-context inputs~\cite{spin,owl,judge_decoding}. Despite strong progress, the dominant implementation remains \emph{linear} drafting, where a single speculative chain is proposed per iteration; when early tokens are rejected, downstream drafted tokens are discarded, causing substantial wasted computation and limiting utilization of parallel verification.

\subsection{Tree-Based and Parallel Decoding}

To overcome the single-path limitation, recent work explores \emph{tree-based} speculative decoding, where multiple candidate continuations are drafted and verified in a single target-model forward pass using structured attention masks. SpecInfer~\cite{specinfer} instantiates this idea in an LLM serving setting by building a token tree and verifying it efficiently. OPT-Tree~\cite{opt_tree} further studies \emph{adaptive} tree construction, selecting tree shapes to maximize expected acceptance length under a fixed verification budget. Alternative verification strategies such as traversal-style verification have also been explored for speculative trees~\cite{traversal_verification}. In parallel, Medusa~\cite{medusa} pursues multi-token generation by augmenting a base model with multiple decoding heads and verifying the induced candidate tree; unlike draft--target speculative decoding, it requires model-specific fine-tuning. Our work follows the draft--target paradigm but focuses on practical tree construction and verification under strict budget constraints, emphasizing dynamic pruning as a lightweight mechanism to stabilize performance.

\subsection{Dynamic Pruning Strategies}

Tree-based methods must contend with the exponential growth of candidates with depth and branching. ProPD~\cite{propd} proposes dynamic token-tree pruning and generation, leveraging early signals to remove low-utility branches before full verification. Cost-aware formulations further model verification overhead and explicitly optimize the trade-off between exploration and target-model computation~\cite{cast}. DySpec~\cite{dyspec} employs greedy, confidence-guided expansion to adapt tree structures online. DynaTree is closely related to these approaches: we adopt a probability-threshold-based pruning rule coupled with an explicit node budget, aiming for a simple, training-free mechanism that is easy to integrate while maintaining strong speedups in practice.


\section{Methodology}
\label{method}

\subsection{Problem Setup and Notation}
Let \(M_T\) denote a target autoregressive language model and \(M_D\) a smaller draft model. Given a prefix (prompt) \(x_{1:t}\), greedy decoding with \(M_T\) produces tokens \(y_{t+1}, y_{t+2}, \dots\) where
\[
y_{i} \;=\; \arg\max_{v \in \mathcal{V}} p_T(v \mid x_{1:i-1}).
\]
Speculative decoding accelerates generation by proposing candidate tokens with \(M_D\) and verifying them with \(M_T\), while preserving the greedy output when the verification rule only commits tokens that match the target greedy predictions.

\subsection{Overview of DynaTree}
DynaTree generalizes linear speculative decoding from a single draft chain to a \emph{draft token tree}. In each iteration, DynaTree performs: (i) \textbf{tree drafting} with \(M_D\) by expanding top-\(B\) candidates up to depth \(D\); (ii) \textbf{parallel verification} of all drafted nodes using a tree attention mask in a single forward pass of \(M_T\); (iii) \textbf{path selection and commit} by greedily selecting the longest path consistent with the target model's greedy predictions; and (iv) \textbf{KV-cache update} for the committed tokens.

\begin{figure}[t]
  \centering
  \includegraphics[width=\linewidth]{../figures/dynatree-v7.png}
  \caption{\textbf{Architecture overview of DynaTree.} Each iteration drafts a candidate token tree with a small \emph{draft model}, prunes unlikely branches under a probability threshold and node budget, serializes the remaining nodes and constructs a \emph{tree causal attention mask}, then verifies all candidates in parallel with a single forward pass of the \emph{target model}. A longest valid path (tokens matching the target model's greedy top-1 predictions) is committed, and the context and KV cache are updated for the next iteration.}
  \label{fig:arch}
\end{figure}

\subsection{Draft Tree Construction with Dynamic Pruning}
We maintain a token tree \(\mathcal{T}=(\mathcal{N}, \mathcal{E})\) whose nodes \(u \in \mathcal{N}\) correspond to drafted tokens. Each node stores: token \(z_u \in \mathcal{V}\), parent \(\pi(u)\), depth \(d(u)\), draft log-probability \(\ell_u = \log p_D(z_u \mid \text{prefix}(\pi(u)))\), and cumulative log-probability \(\bar{\ell}_u = \sum_{v \in \text{path}(u)} \ell_v\).

\paragraph{Expansion.}
Starting from the current prefix \(x_{1:t}\), we first obtain the draft distribution \(p_D(\cdot \mid x_{1:t})\) and create the root node \(u_0\) using the top-1 token. Then, for each active leaf \(u\) with \(d(u) < D\), we expand children by selecting the top-\(B\) tokens under \(p_D(\cdot \mid x_{1:t+\text{pos}(u)})\) (implemented via cached one-token forward passes). To bound computation, we enforce a hard \textbf{node budget} \(N_{\max}\) and stop expansion when \(|\mathcal{N}| \ge N_{\max}\).

\paragraph{Probability-threshold pruning (V2).}
To avoid wasting draft computation on unlikely branches, DynaTree prunes any leaf \(u\) whose cumulative probability falls below a threshold \(\tau \in (0,1)\):
\[
\bar{\ell}_u \;<\; \log \tau \quad \Rightarrow \quad \text{prune } u.
\]
This rule corresponds exactly to the implementation in `TreeSpeculativeGeneratorV2`, where \(\bar{\ell}_u\) is accumulated along the path and compared against \(\log\tau\).

\begin{algorithm}[t]
\caption{\textbf{DynaTree: one iteration (greedy-consistent).}}
\label{fig:algo}
\KwIn{Prefix tokens \(x_{1:t}\); target KV cache \(\mathcal{K}_T\); prefix next-token logits \(\mathbf{s}_{\text{last}}\);
tree depth \(D\); branch factor \(B\); pruning threshold \(\tau\); node budget \(N_{\max}\).}
\KwOut{Committed tokens \(y_{t+1:t+L}\) and updated target cache \(\mathcal{K}_T\).}

\BlankLine
\Comment*[l]{Main procedure (function-wrapped, SpecInfer-style)}
\(\ell \leftarrow \textsc{SeqLen}(\mathcal{K}_T)\)\;
\(\mathcal{T} \leftarrow \FDraft(x_{1:t}, D, B, \tau, N_{\max})\)\;
\((\mathbf{z}_{1:n}, \mathbf{A}) \leftarrow \FFlatten(\mathcal{T}, \ell)\)\;
\(\hat{\mathbf{y}} \leftarrow \FVerify(\mathbf{z}_{1:n}, \mathbf{A}, \mathcal{K}_T)\)\;
\(y_{t+1:t+L} \leftarrow \FSelect(\mathcal{T}, \hat{\mathbf{y}}, \mathbf{s}_{\text{last}})\)\;
\(\mathcal{K}_T \leftarrow \FUpdate(\mathcal{K}_T, y_{t+1:t+L}, \ell)\)\;
\Return \(y_{t+1:t+L}\)\;

\BlankLine
\Fn{\FDraft(\(x_{1:t}, D, B, \tau, N_{\max}\))}{
  \Comment*[l]{Draft a token tree with V2 pruning (matches \texttt{TreeSpeculativeGeneratorV2})}
  Run \(M_D\) on \(x_{1:t}\); add the \(\top 1\) token as root \(u_0\)\;
  \(\mathcal{A}\leftarrow\{u_0\}\)\;
  \For{\(d\leftarrow 1\) \KwTo \(D\)}{
    \If{\(|\mathcal{A}|=0\) \textbf{or} \(|\mathcal{T}|\ge N_{\max}\)}{\textbf{break}}
    \(\mathcal{A}'\leftarrow\emptyset\)\;
    \ForEach{\(u\in\mathcal{A}\)}{
      \If{\(\bar{\ell}_u < \log\tau\)}{\textbf{continue}} \Comment*[f]{skip expanding low-prob branches}
      Do one cached step of \(M_D\) from \(u\); take \(\top B\) next-token candidates\;
      Add up to \(B\) children (stop if \(|\mathcal{T}|=N_{\max}\)); add children to \(\mathcal{A}'\)\;
    }
    \(\mathcal{A}\leftarrow\mathcal{A}'\)\;
  }
  \Return \(\mathcal{T}\)\;
}

\BlankLine
\Fn{\FFlatten(\(\mathcal{T}, \ell\))}{
  \Comment*[l]{BFS serialization + tree causal mask}
  \(\mathbf{z}_{1:n} \leftarrow \textsc{BFSFlatten}(\mathcal{T})\)\;
  \(\mathbf{A} \leftarrow \textsc{TreeMask}(\mathcal{T}, \text{prefix\_len}=\ell)\)\;
  \Return \((\mathbf{z}_{1:n}, \mathbf{A})\)\;
}

\BlankLine
\Fn{\FVerify(\(\mathbf{z}_{1:n}, \mathbf{A}, \mathcal{K}_T\))}{
  \Comment*[l]{Single target forward pass with tree attention}
  \(\mathbf{s}_{1:n} \leftarrow M_T(\mathbf{z}_{1:n}; \mathbf{A}, \mathcal{K}_T)\)\;
  \(\hat{\mathbf{y}} \leftarrow \arg\max \mathbf{s}_{1:n}\)\;
  \Return \(\hat{\mathbf{y}}\)\;
}

\BlankLine
\Fn{\FSelect(\(\mathcal{T}, \hat{\mathbf{y}}, \mathbf{s}_{\text{last}}\))}{
  \Comment*[l]{Longest greedy-consistent path (matches \texttt{\_select\_best\_path})}
  \(first \leftarrow \arg\max \mathbf{s}_{\text{last}}\)\;
  Find the longest path \(P\) starting at the root such that:
  root token \(=\) \(first\), and for each edge \((u\!\rightarrow\!v)\), token\((v)=\hat{\mathbf{y}}[\text{pos}(u)]\)\;
  \eIf{\(P=\emptyset\)}{
    \Return \([first]\)\;
  }{
    \(y \leftarrow\) tokens on \(P\)\;
    Append one bonus token \(=\hat{\mathbf{y}}[\text{pos}(\text{last}(P))]\)\;
    \Return \(y\)\;
  }
}

\BlankLine
\Fn{\FUpdate(\(\mathcal{K}_T, y_{t+1:t+L}, \ell\))}{
  \Comment*[l]{Rollback-and-rebuild cache (matches \texttt{\_update\_tree\_cache})}
  \(\mathcal{K}_T \leftarrow \textsc{Crop}(\mathcal{K}_T,\ell)\)\;
  \(\mathcal{K}_T \leftarrow M_T(y_{t+1:t+L}; \mathcal{K}_T)\)\;
  \Return \(\mathcal{K}_T\)\;
}
\end{algorithm}

\subsection{Tree Attention for Parallel Verification}
To verify all drafted tokens in one target-model forward pass, we \emph{flatten} the tree in breadth-first order (BFS), producing a sequence \(z_{1:n}\) where each token corresponds to one node and all ancestors appear earlier than descendants. We then construct a boolean attention mask \(\mathbf{A} \in \{0,1\}^{n \times (t+n)}\) such that each drafted token attends to: (i) all prefix tokens \(x_{1:t}\), and (ii) only its ancestors (including itself) in the flattened tree:
\[
\mathbf{A}_{i,j} =
\begin{cases}
1, & 1 \le j \le t,\\
1, & j=t+\mathrm{pos}(v) \text{ for some ancestor } v \in \mathrm{Anc}(u_i)\cup\{u_i\},\\
0, & \text{otherwise.}
\end{cases}
\]
This mask ensures the conditional distribution computed at each node matches the distribution of sequential decoding along its unique root-to-node path, while enabling parallel verification across different branches~\cite{specinfer,opt_tree}.

\begin{figure}[t]
  \centering
  \fbox{\rule[-.5cm]{0cm}{3.0cm}\rule[-.5cm]{0.95\linewidth}{0cm}}
  \caption{\textbf{Tree attention mask (placeholder).} Example tree, BFS flattening, and the induced causal mask where each node attends only to the prefix and its ancestors.}
  \label{fig:tree-attn}
\end{figure}

\subsection{Greedy Path Selection and Cache Update}
\paragraph{Verification signals.}
Let \(\hat{y}_{t+1} = \arg\max p_T(\cdot \mid x_{1:t})\) be the target model's greedy next token from the prefix (available from the prefix logits). For each tree node \(u\) with flattened position \(i\), the target forward pass outputs logits \(\mathbf{s}_i\), whose argmax \(\hat{y}(u)=\arg\max \mathbf{s}_i\) corresponds to the greedy \emph{next-token} prediction after consuming the path to \(u\).

\paragraph{Longest valid path.}
DynaTree commits the longest path \(u_0 \rightarrow u_1 \rightarrow \cdots \rightarrow u_m\) such that the drafted token at each node matches the target greedy prediction from its parent context:
\[
z_{u_0}=\hat{y}_{t+1},\quad
z_{u_{k}}=\hat{y}(u_{k-1}) \;\; \text{for } k=1,\dots,m.
\]
If no drafted token matches the first greedy prediction, we fall back to committing \(\hat{y}_{t+1}\) (one token progress). After committing the matched draft tokens, we append one \emph{bonus} token \(\hat{y}(u_m)\) from the target model, mirroring the greedy speculative decoding convention and ensuring steady progress.

\paragraph{KV-cache management.}
Tree verification may populate KV states for non-committed branches. To maintain correctness, we crop the target cache back to the pre-iteration prefix length \(t\) and then forward only the committed tokens to rebuild the cache (as implemented in `_update_tree_cache`).

\subsection{Correctness for Greedy Decoding}
We sketch the correctness argument for greedy decoding (the setting used throughout our experiments). The tree attention mask guarantees that for any node \(u\), the target logits at \(u\) are computed from exactly the same conditioning context as in sequential decoding along the root-to-\(u\) path. DynaTree commits a drafted token \emph{only if} it equals the target greedy argmax under that context. Therefore, every committed token matches the token that greedy decoding with \(M_T\) would produce at that position. The cache rollback-and-rebuild step ensures the subsequent iteration starts from an identical KV state. Consequently, DynaTree generates exactly the same token sequence as greedy decoding with the target model, while reducing the number of expensive target-model forward passes by verifying many candidate tokens in parallel.

\subsection{Complexity Discussion}
Let \(n=|\mathcal{N}|\le N_{\max}\) be the number of drafted nodes. Drafting requires \(O(n)\) one-token forward passes of the draft model (with cache reuse across expansions). Verification requires a single target-model forward pass over \(n\) tokens with a structured attention mask. Dynamic pruning reduces \(n\) in uncertain regions by discarding low-probability branches, improving the trade-off between draft overhead and verification parallelism.


\section{Experiments}
\label{experiments}

\subsection{Experimental Setup}
\paragraph{Models.}
We evaluate on the Pythia family with a target model \(M_T=\) Pythia-2.8B and a draft model \(M_D=\) Pythia-70M, using greedy decoding throughout (\texttt{do\_sample=False}).

\paragraph{Hardware and software.}
All experiments are run on a single NVIDIA GPU. We use PyTorch 2.x and HuggingFace Transformers 4.x with \texttt{DynamicCache} for KV management.

\paragraph{Workloads.}
Unless otherwise stated, we generate 500 new tokens from a fixed technical prompt and report averages over 5 runs, skipping the first run as warmup. For each run, we synchronize the GPU before timing and clear caches between methods to reduce cross-run interference.

\subsection{Metrics}
We report \textbf{throughput} (tokens/sec) as the primary metric. We also report \textbf{speedup} relative to the autoregressive baseline, and (when available) \textbf{acceptance rate} and \textbf{tokens per iteration} (average committed tokens per verification step).

\subsection{Baselines}
We compare against:
(i) \textbf{AR} greedy decoding with the target model;
(ii) \textbf{HuggingFace assisted generation} (built-in speculative decoding using \texttt{assistant\_model});
(iii) \textbf{linear speculative decoding} implemented with a draft chain of length \(K\);
and (iv) \textbf{StreamingLLM + speculative decoding} for long-context cache compression~\cite{streamingllm}.

\subsection{Main Results}
Table~\ref{tab:main-results} summarizes end-to-end throughput on 500-token generation. DynaTree (Tree V2) achieves the best speedup (1.62\(\times\)) and outperforms both HuggingFace assisted generation (1.36\(\times\)) and linear speculative decoding (1.11\(\times\)). We additionally report the draft-token acceptance rate when available.

\begin{table}[t]
\centering
\caption{Main decoding throughput on Pythia (500-token generation). Higher is better.}
\label{tab:main-results}
\begin{tabular}{lcccc}
\toprule
Method & Throughput (t/s) & Speedup & Accept. & Notes \\
\midrule
AR (target-only) & 119.4 & 1.00 & -- & Greedy decoding \\
HuggingFace assisted & 161.9 & 1.36 & -- & \texttt{assistant\_model} \\
Linear speculative (K=6) & 133.1 & 1.11 & 0.68 & Draft-chain verification \\
StreamingLLM + speculative (cache=1024) & 132.9 & 1.11 & -- & Cache compression \\
\textbf{DynaTree (Tree V2)} & \textbf{193.4} & \textbf{1.62} & 0.30 & \(D{=}8,B{=}3,\tau{=}0.03,N_{\max}{=}128\) \\
\bottomrule
\end{tabular}
\end{table}

Although the acceptance rate of tree-based drafting is lower than that of linear drafting, DynaTree improves throughput by increasing the probability of finding a long prefix consistent with the target greedy trajectory within a single verification step, while pruning low-probability branches to keep the verification set compact.

\begin{figure}[t]
  \centering
  \fbox{\rule[-.5cm]{0cm}{3.2cm}\rule[-.5cm]{0.95\linewidth}{0cm}}
  \caption{\textbf{Main results visualization (placeholder).} Bar chart of throughput/speedup for Table~\ref{tab:main-results}.}
  \label{fig:main-results}
\end{figure}

\subsection{Ablation Study: Progressive Component Addition}
To isolate the contribution of tree structure and pruning/optimization, we use the exhaustive sweep to form a progressive ablation sequence (Linear \(\rightarrow\) Tree \(\rightarrow\) Optimized Tree). Table~\ref{tab:ablation} reports throughput and speedup under the sweep harness. Although absolute throughput can differ from the end-to-end benchmark due to different measurement harnesses and warmup effects, the progressive improvements highlight the algorithmic value of (i) parallel path verification and (ii) deeper trees with calibrated pruning.

\begin{table}[t]
\centering
\caption{Ablation study (progressive component addition) extracted from the sweep results.}
\label{tab:ablation}
\begin{tabular}{lccc}
    \toprule
Method & Configuration & Throughput (t/s) & Speedup \\
    \midrule
Linear speculative & K=6 & 133.1 & 1.11 \\
+ Tree structure & \(D{=}4,B{=}3,\tau{=}0.01\) & 176.6 & 1.43 \\
\textbf{+ Depth \& pruning optimization} & \textbf{\(D{=}8,B{=}3,\tau{=}0.03\)} & \textbf{221.4} & \textbf{1.79} \\
    \bottomrule
  \end{tabular}
\end{table}

\subsection{Hyperparameter Sensitivity (Summary)}
We conduct an extensive sweep over tree depth \(D\), branching factor \(B\), and pruning threshold \(\tau\) across multiple generation lengths (100--1000 tokens), totaling 450 configurations. For clarity and space, we place the detailed sweep plots in Appendix~\ref{app:sweep} and summarize the key trend: increasing depth and branching improves exploration but can reduce throughput if verification cost grows faster than the expected committed length; probability-threshold pruning is essential to keep the effective tree size within budget.

\subsection{Sequence Length Scaling}
Table~\ref{tab:length-scaling} reports the best-performing Tree V2 configuration selected per generation length, together with baseline throughput and speedup. We observe that speedup increases from short to medium lengths as verification overhead is amortized, peaking around 500 tokens in our setting.

\begin{table}[t]
\centering
\caption{Best Tree V2 performance across different generation lengths.}
\label{tab:length-scaling}
\begin{tabular}{lccccc}
\toprule
Length & Optimal \((D,B,\tau)\) & Baseline (t/s) & Tree V2 (t/s) & Speedup & Accept. \\
\midrule
100  & (7,3,0.03) & 105.3 & 150.7 & 1.43 & 0.31 \\
200  & (7,3,0.03) & 125.5 & 193.2 & 1.54 & 0.36 \\
300  & (7,3,0.03) & 124.2 & 199.0 & 1.60 & 0.38 \\
500  & (8,3,0.03) & 123.9 & 221.4 & 1.79 & 0.38 \\
1000 & (6,3,0.05) & 124.5 & 212.3 & 1.71 & 0.37 \\
\bottomrule
\end{tabular}
\end{table}


\section{Conclusion}
We introduced DynaTree, a tree-based speculative decoding framework that drafts multiple candidate continuations and verifies them in parallel using tree attention, while controlling verification cost via probability-threshold pruning and an explicit node budget. Across Pythia models, DynaTree improves decoding throughput over autoregressive decoding and consistently outperforms strong speculative decoding baselines. Our results suggest that multi-branch exploration, coupled with lightweight pruning, is an effective way to better utilize target-model verification compute under strict budget constraints. A key direction for future work is improving robustness across diverse prompts and long-context settings, and reducing overhead via kernel-level optimizations and hardware-aware tree construction.

\appendix
\section{Hyperparameter Sweep Details}
\label{app:sweep}
We perform a grid search over tree depth $D \in \{3,4,5,6,7,8\}$, branching factor $B \in \{2,3,4\}$, and pruning threshold $\tau \in \{0.01,0.02,0.03,0.05,0.1\}$ across generation lengths 100--1000, totaling 450 configurations. Figure~\ref{fig:param-sweep} visualizes the depth--breadth--threshold trade-off, together with the relationship between effective tree size and speedup.

\begin{figure}[t]
  \centering
  \fbox{\rule[-.5cm]{0cm}{3.2cm}\rule[-.5cm]{0.95\linewidth}{0cm}}
  \caption{\textbf{Parameter sweep (placeholder).} Suggested 6-panel plot: speedup vs depth, vs branching factor, vs threshold; speedup vs length; average tree size heatmap; acceptance/tokens-per-round distribution.}
  \label{fig:param-sweep}
\end{figure}

\bibliographystyle{unsrtnat}
\bibliography{references}

\end{document}